\documentclass[a4paper]{article}
\usepackage[spanish]{babel}
\usepackage[utf8]{inputenc}
\usepackage[usenames]{color}
\usepackage{charter}   % tipografia
\usepackage{graphicx}
\usepackage{enumerate}
\usepackage{listings}
\usepackage{color}
%\usepackage{indentfirst}
\usepackage{fancyhdr}
\usepackage{verbatim}
\usepackage{latexsym}
\usepackage{lastpage}
\usepackage[colorlinks=true, linkcolor=black]{hyperref}
%\usepackage{makeidx}
%\usepackage{float}
\usepackage{calc}
\usepackage{amsthm, amssymb}
\usepackage[nosumlimits]{amsmath} % este package hace que se vean mal los incides en las sumatorias, pero permite poner uno abajo del otro en la ecuacon de los L de laagrange

\usepackage{subfig}

\usepackage{amsfonts}
\definecolor{gray}{gray}{0.5}
\definecolor{light-gray}{gray}{0.95}
\definecolor{orange}{rgb}{1,0.5,0}

\input{codesnippet}
\input{page.layout}
\usepackage{underscore}
\usepackage{caratulaV}
\usepackage{url}
\usepackage{float}

\usepackage{underscore}
\usepackage{alltt}
\usepackage{tikz}
\usepackage{color}
\usepackage{verbatim}
\usepackage{algorithm}
\usepackage{algpseudocode}

\definecolor{dkgreen}{rgb}{0,0.6,0}
\definecolor{gray}{rgb}{0.5,0.5,0.5}
\definecolor{mauve}{rgb}{0.58,0,0.82}

\lstset{frame=tb,
  language=Python,
  aboveskip=3mm,
  belowskip=3mm,
  showstringspaces=false,
  columns=flexible,
  basicstyle={\small\ttfamily},
  keywordstyle=\color{blue},
  commentstyle=\color{dkgreen},
  stringstyle=\color{mauve},
  breaklines=true,
  breakatwhitespace=true,
  tabsize=3,
  numbers=left,
  xleftmargin=2em,
  frame=single,
  framexleftmargin=2em,
  numbersep=5pt,                   % how far the line-numbers are from the code
  numberstyle=\small\color{gray} % the style that is used for the line-numbers
 }

\parskip = 5 pt

\newcounter{row}
\newcounter{col}

\newcommand\setrow[3]{
	\setcounter{col}{1}
	\foreach \n in {#1, #2, #3} {
	\edef\x{\value{col} - 0.5}
	\edef\y{3.5 - \value{row}}
	\node[anchor=center] at (\x, \y) {\n};
	\stepcounter{col}
	}
	\stepcounter{row}
}



\newcommand\setrowaux[7]{
	\setcounter{col}{1}
	\foreach \n in {#1, #2, #3, #4, #5, #6, #7} {
	\edef\x{\value{col} - 0.5}
	\edef\y{7.5 - \value{row}}
	\node[anchor=center] at (\x, \y) {\n};
	\stepcounter{col}
	}
	\stepcounter{row}
}

\newcommand\setrowauxx[4]{
	\setcounter{col}{1}
	\foreach \n in {#1, #2, #3, #4} {
	\edef\x{\value{col} - 0.5}
	\edef\y{4.5 - \value{row}}
	\node[anchor=center] at (\x, \y) {\n};
	\stepcounter{col}
	}
	\stepcounter{row}
}


\begin{document}


\parskip = 5 pt
\thispagestyle{empty}
\materia{Aprendizaje Automático}
\titulo{Trabajo Práctico 1}
\integrante{Gustavo Cairo}{Xxx}{gj.cairo@gmail.com}
\integrante{Germán Pinzón}{475/13}{pinzon.german.94@gmail.com}
\integrante{Ángel More}{931/12}{angel\_21\_fer@hotmail.com}

\maketitle




\newpage
\tableofcontents
\thispagestyle{empty}

\newpage
\section{Introducción:}





\newpage

\section{Desarrollo:}

\subsection{Extracción de atributos:}

Se realizaron un total de 505 extracciones de atributos sobre, aproximadamente, 70K de mails (nuestro set train). 

Los atributos seleccionados fueron los siguientes:
\begin{itemize}
\item Len : Se toma como atributo la longitud de cada mail.
\item count\_spaces : representa la cantidad de espacios que contienen los mails.
\item links : Cantidad de enlaces https que contienen.
\item tags : cantidad de tags html encontrados.
\item rare : Este atributo representa la cantidad de caracteres, que a nuestro criterio, serian poco comunes encontrar en un mail del tipo Ham.
\item word\_count\_att\_names : Para este caso se extrajeron las 500 palabras mas utilizadas del total de mails utilizados como base de entrenamiento. \\
Para la extracción se exceptuó tanto el encabezado de cada mail como la porción de texto que corresponde a mensajes reenviados. Tomamos estas decisiones porque de esta manera estamos excluyendo palabras que usualmente pueden estar presentes sin importar el tipo de mail, y por lo que considerarlas no nos brindaría ningún tipo información representativa a la hora de querer clasificarlos.     

\end{itemize}



\subsection{Modelos:}





\subsection{Reducción de dimensionalidad:}



\newpage

\subsection{Resultados:}



\newpage

\subsection{Discusión:}



\newpage

\section{Conclusión:}
\end{document}