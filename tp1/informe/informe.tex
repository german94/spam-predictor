\documentclass[a4paper]{article}
\usepackage[spanish]{babel}
\usepackage[utf8]{inputenc}
\usepackage[usenames]{color}
\usepackage{charter}   % tipografia
\usepackage{graphicx}
\usepackage{enumerate}
\usepackage{listings}
\usepackage{color}
%\usepackage{indentfirst}
\usepackage{fancyhdr}
\usepackage{verbatim}
\usepackage{latexsym}
\usepackage{lastpage}
\usepackage[colorlinks=true, linkcolor=black]{hyperref}
%\usepackage{makeidx}
%\usepackage{float}
\usepackage{calc}
\usepackage{amsthm, amssymb}
\usepackage[nosumlimits]{amsmath} % este package hace que se vean mal los incides en las sumatorias, pero permite poner uno abajo del otro en la ecuacon de los L de laagrange

\usepackage{subfig}

\usepackage{amsfonts}
\definecolor{gray}{gray}{0.5}
\definecolor{light-gray}{gray}{0.95}
\definecolor{orange}{rgb}{1,0.5,0}

\usepackage{color} % para snipets de codigo coloreados
\usepackage{fancybox}  % para el sbox de los snipets de codigo

\definecolor{litegrey}{gray}{0.94}

% \newenvironment{sidebar}{%
% 	\begin{Sbox}\begin{minipage}{.85\textwidth}}%
% 	{\end{minipage}\end{Sbox}%
% 		\begin{center}\setlength{\fboxsep}{6pt}%
% 		\shadowbox{\TheSbox}\end{center}}
% \newenvironment{warning}{%
% 	\begin{Sbox}\begin{minipage}{.85\textwidth}\sffamily\lite\small\RaggedRight}%
% 	{\end{minipage}\end{Sbox}%
% 		\begin{center}\setlength{\fboxsep}{6pt}%
% 		\colorbox{litegrey}{\TheSbox}\end{center}}

\newenvironment{codesnippet}{%
	\begin{Sbox}\begin{minipage}{\textwidth}\sffamily\small}%
	{\end{minipage}\end{Sbox}%
		\begin{center}%
		\colorbox{litegrey}{\TheSbox}\end{center}}



\usepackage{fancyhdr}
\pagestyle{fancy}

%\renewcommand{\chaptermark}[1]{\markboth{#1}{}}
\renewcommand{\sectionmark}[1]{\markright{\thesection\ - #1}}

\fancyhf{}

\fancyhead[LO]{Sección \rightmark} % \thesection\ 
\fancyfoot[LO]{\small{Gustavo Cairo, Germán Pinzón, Ángel More}}
\fancyfoot[RO]{\thepage}
\renewcommand{\headrulewidth}{0.5pt}
\renewcommand{\footrulewidth}{0.5pt}
\setlength{\hoffset}{-0.8in}
\setlength{\textwidth}{16cm}
%\setlength{\hoffset}{-1.1cm}
%\setlength{\textwidth}{16cm}
\setlength{\headsep}{0.5cm}
\setlength{\textheight}{25cm}
\setlength{\voffset}{-0.7in}
\setlength{\headwidth}{\textwidth}
\setlength{\headheight}{13.1pt}

\renewcommand{\baselinestretch}{1.1}  % line spacing


\usepackage{underscore}
\usepackage{caratulaV}
\usepackage{url}
\usepackage{float}

\usepackage{underscore}
\usepackage{alltt}
\usepackage{tikz}
\usepackage{color}
\usepackage{verbatim}
\usepackage{algorithm}
\usepackage{algpseudocode}

\definecolor{dkgreen}{rgb}{0,0.6,0}
\definecolor{gray}{rgb}{0.5,0.5,0.5}
\definecolor{mauve}{rgb}{0.58,0,0.82}

\lstset{frame=tb,
  language=Python,
  aboveskip=3mm,
  belowskip=3mm,
  showstringspaces=false,
  columns=flexible,
  basicstyle={\small\ttfamily},
  keywordstyle=\color{blue},
  commentstyle=\color{dkgreen},
  stringstyle=\color{mauve},
  breaklines=true,
  breakatwhitespace=true,
  tabsize=3,
  numbers=left,
  xleftmargin=2em,
  frame=single,
  framexleftmargin=2em,
  numbersep=5pt,                   % how far the line-numbers are from the code
  numberstyle=\small\color{gray} % the style that is used for the line-numbers
 }

\parskip = 5 pt

\newcounter{row}
\newcounter{col}

\newcommand\setrow[3]{
	\setcounter{col}{1}
	\foreach \n in {#1, #2, #3} {
	\edef\x{\value{col} - 0.5}
	\edef\y{3.5 - \value{row}}
	\node[anchor=center] at (\x, \y) {\n};
	\stepcounter{col}
	}
	\stepcounter{row}
}



\newcommand\setrowaux[7]{
	\setcounter{col}{1}
	\foreach \n in {#1, #2, #3, #4, #5, #6, #7} {
	\edef\x{\value{col} - 0.5}
	\edef\y{7.5 - \value{row}}
	\node[anchor=center] at (\x, \y) {\n};
	\stepcounter{col}
	}
	\stepcounter{row}
}

\newcommand\setrowauxx[4]{
	\setcounter{col}{1}
	\foreach \n in {#1, #2, #3, #4} {
	\edef\x{\value{col} - 0.5}
	\edef\y{4.5 - \value{row}}
	\node[anchor=center] at (\x, \y) {\n};
	\stepcounter{col}
	}
	\stepcounter{row}
}


\begin{document}


\parskip = 5 pt
\thispagestyle{empty}
\materia{Aprendizaje Automático}
\titulo{Trabajo Práctico 1}
\integrante{Gustavo Cairo}{Xxx}{gj.cairo@gmail.com}
\integrante{Germán Pinzón}{475/13}{pinzon.german.94@gmail.com}
\integrante{Ángel More}{931/12}{angel\_21\_fer@hotmail.com}

\maketitle




\newpage
\tableofcontents
\thispagestyle{empty}

\newpage
\section{Introducción:}

El presente trabajo tiene como objetivo elaborar un clasificador de mails en Spam y Ham. Para lograr esto trabajaremos con distintos modelos de predicción. 


Dado que los modelos necesitan de ciertos atributos para poder analizar, catalogar y así predecir , extraeremos un conjunto de atributos significativos de nuestro conjunto de mail.


Trabajaremos sobre un total de 90K de mails, de los cuales utilizaremos 70K como base de entrenamiento para los modelos y el resto como test set. 
 Evaluaremos distintas combinaciones y variaciones dentro de los modelos comparando la exactitud de cada uno así como el costo temporal que estos implican. 
 
 Finalmente seleccionaremos los mejores y los utilizaremos para clasificar a los mails perteneciente al test set y de esta manera tener una aproximación de como se comportaría el clasificador frente a nuevos datos.  
 
 
El código será elaborado en lenguaje python. Como bibliotecas destacadas utilizaremos $Pandas$, esta nos permitirá el almacenamiento de los mails y sus atributos dentro de la estructura $DataFrame$.

Las funciones que representen a los distintos modelos serán provistas por la biblioteca $sklearn$.



\newpage

\section{Desarrollo:}

\subsection{Extracción de atributos:}

Para comenzar con la construcción de los modelos realizaremos la extracción de un conjunto de atributos que creemos pueden permitirnos diferenciar un mail Spam de uno Ham. Realizamos un total de 505 extracciones sobre 70K de mails (aproximadamente). Dicho conjunto de mail sera nuestro train set mientras que otros 20K se corresponderan a nuestro test set.

Los atributos seleccionados fueron los siguientes:
\begin{itemize}
\item Len : Se toma como atributo la longitud de cada mail.
\item count\_spaces : representa la cantidad de espacios que contienen los mails.
\item links : Cantidad de enlaces https que contienen. Ya que suponemos que un tipo de mail spam generalmente es aquel con publicidades y estos con tienen distintos enlaces a sus diversos servicios comerciales; o así también haciendo referencia a  noticias.
\item tags : cantidad de tags html encontrados. A veces no solo podemos encontrar en los mail enlaces como los que describimos para el atributo "links"; sino que la imágenes, publicidades, noticias, etc se pueden visualizar dentro del propio mail por lo que la presencia y/o cantidad de etiquetas html deberían ser mayor en los spam. De esta manera constituiría otro atributo distintivo. 
\item rare : Este atributo representa la cantidad de caracteres, que a nuestro criterio, serian poco comunes encontrar en un mail del tipo Ham.
\item word\_count\_att\_names : Para este caso se extrajeron las 500 palabras mas utilizadas del total de mails utilizados como base de entrenamiento. \\
Para la extracción se exceptuó tanto el encabezado de cada mail como la porción de texto que corresponde a mensajes reenviados. Tomamos estas decisiones porque de esta manera estamos excluyendo palabras que usualmente pueden estar presentes sin importar el tipo de mail, y por lo que considerarlas no nos brindaría ningún tipo de información representativa a la hora de querer clasificarlos.   
  

\end{itemize}



\subsection{Modelos:}





\subsection{Reducción de dimensionalidad:}
Las técnicas que decidimos emplear para reducir la dimensionalidad fueron básicamente PCA, selección de $K$ mejores atributos y selección de percentil. Estas tres técnicas fueron combinadas siempre que fue posible y probadas con distintos parámetros en el caso de las últimas dos.

Para el caso de selección de $K$ mejores atributos variamos el K con los siguientes valores: 20, 50, 100, 150, 200, 250, 300, 320, 370, 400, 420 y 450 ya que pensamos que constituían una gama relativamente amplia de valores posibles, considerando que estamos trabajando con un total de 505 atributos. En el caso de la selección de percentil variamos el parámetro percentil con los valores 5, 10, 15 y 25. Además, dado que en algunos casos decidimos hacer una combinación de técnicas aplicando primero $PCA$ y luego alguna de las otras dos, también utilizamos dos funciones distintas en el parámetro $score\_func$ correspondiente a $SelectKBest$ y $SelectPercentile$.

Las funciones que utilizamos en el parámetro $score\_func$ fueron $chi2$ y $f\_classif$. Cuando decidimos aplicar primero $PCA$ y luego algún tipo de selección entonces tuvimos que utilizar la función $f\_classif$ debido a que si no se presentaban inconvenientes por valores negativos que aparecían en la matriz luego de la transformación que aplicaba $PCA$. Sin embargo cuando aplicamos las funciones de selección por separado, sin combinarlas con $PCA$, pudimos experimentar tanto con $chi2$ como con $f\_classif$.

Un aspecto importante a rescatar es que tuvimos que hacer utilización de la clase $Pipeline$ de sklearn. Esto se debe a que como vamos a estar realizando ciertas transformaciones a los datos con los que nuestro clasificador vaya a entrenar, es necesario que dichas transformaciones se realicen en el momento indicado que es dentro de la función $cross\_val\_score$ (que usamos para medir la performance de nuestro clasificador en esta instancia). Así, si por ejemplo decidimos aplicar primero $PCA$ y luego $SelectKBest$ con $chi2$ y $K = 200$, entonces estas dos funciones (con parámetros en el caso de la segunda) van a ser steps en el pipeline que utilizará $cross\_val\_score$.

Finalmente para poder tener cierta variedad de transformaciones posibles en los datos a la hora de aplicar los clasificadores y decidir con cual quedarnos, vamos a estar aplicando lo siguiente:
\begin{itemize}
\item $PCA$
\item $SelectKBest$ con los valores de $K$ mencionados anteriormente y el parámetro $score\_func$ tanto con $f\_classif$ como con $chi2$
\item $SelectPercentile$ con los valores de $percentile$ mencionados anteriormente
\item $PCA$ y luego $SelectKBest$ (mismos valores de $K$ pero con $f\_classif$ en $score\_func$)
\item $PCA$ y luego $SelectPercentile$ (mismos valores de $percentile$)
\end{itemize}

Luego de hacer las transformaciones anteriores a los datos se aplicará cada uno de los clasificadores elegidos a excepción de Naive Bayes Multinomial en los casos donde se utilice $PCA$ ya que este clasificador no puede tratar ciertos valores negativos que aparecen en la matriz luego de la transformación.

\newpage

\subsection{Resultados y selección del mejor clasificador:}
Para la selección del mejor clasificador lo que hicimos fue recopilar tres tipos de información distinta. Luego de realizar Grid Search y buscar los mejores parámetros dentro de cierto rango para cada clasificador (a excepción de $SVM$ por su complejidad tempral y $GaussianNB$ porque no recibe parámetros) nos quedamos con los datos de accuracy que arrojados por $cross\_val\_score$ (siempre sobre los datos de training). En esta etapa ya es notorio que los clasificadores más performantes son RandomForest y DecisionTree bien en lo alto y luego KNearestNeighbour y por último y bastante alejado Naive Bayes Multinomial (se mostrarán los gráficos con los valores específicos más abajo).

Por otro lado, como dijimos en la sección donde explicamos las transformaciones que realizamos sobre los datos para reducir la dimensionalidad, también aplicamos $cross\_val\_score$ y nos quedamos con el accuracy que arroja esta función sobre los datos transformados por cada una de las técnicas (o combinación de técnicas) con cada clasificador.

Una vez hecho todo lo anterior, vamos a proceder a elegir la mejor configuración de datos que recibe cada clasificador. Es decir, en algunos casos una reducción de dimensionalidad pudo resultar más beneficiosa y en otros no, con lo cual para cada clasificador decidimos si vamos a transformar los datos antes de utilizarlo (y qué tipo de transformación vamos a estar realizando) o no. Si bien más abajo se mostrarán los resultados de una manera gráfica y precisa, por lo general se dió que aplicar alguna transformación a los datos siempre va a resultar mejor que no hacerlo. Una vez hecho esto y seleccionados los clasificadores con sus correspondientes transformaciones, los aplicamos sobre los datos de test, que es la primera vez que los utilizamos. Esto nos brinda una visión mucho más realista de la performance de nuestros clasificadores y nos da cierta seguridad de no haber estado haciendo sobreajuste ya que, como mencionamos, estos datos fueron separados desde el principio y los clasificadores no fueron entrenados con ellos. Más aún, estos 20.000 mails de test que utilizamos fueron extraídos de manera aleatoria utilizando la función $random.sample$.


\newpage

\subsection{Discusión:}



\newpage

\section{Conclusión:}
\end{document}